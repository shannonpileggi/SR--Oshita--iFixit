\documentclass[12pt]{article}
\usepackage{natbib}  % used for citations
\usepackage[parfill]{parskip} %used for formatting style of text

% Sets margins to 1 in
\addtolength{\oddsidemargin}{-.5in}%
\addtolength{\evensidemargin}{-.5in}%
\addtolength{\textwidth}{1in}%
\addtolength{\textheight}{1.3in}%
\addtolength{\topmargin}{-.8in}%


%-----------------------------------------------------------------------------------
\title{Rough Draft}
\author{Lisa Oshita}
\date{}
%-----------------------------------------------------------------------------------


\begin{document}

\maketitle

\section{Abstract}

Community-driven online question and answer forums (CQA) are becoming increasingly valuable sources of information. These platforms house an expansive amount of crowd-sourced knowledge in the form of thousands of questions and answers posted everyday. There are forums that cover a broad range of topics, like Yahoo! Answers, and forums focused on specific topics, like computer programming-focused Stack Overflow. An example of the latter is iFixit's CQA, Answers. As the main goal of iFixit is to teach people to fix their own devices by providing them with the knowledge and tools necessary, Answers---featuring questions related specifically to device repair---is a valuable source of information for this community. Thus, it's important that questions receive answers, and receive them quickly. This paper presents a survival analysis on the time until a question recieves its first answer. We developed a cox proportional hazards model to predict the failure probability of a question, or the probability that a question receives an answer before a certain time. Though our model was signficant, its predictive accuracy was low. Our findings indicate that the most important predictors were the device category of the question (questions pertaining to Apple products receiving answers faster than others), and factors related to the question's title (e.g., whether or not it's phrased as a question, or if it contains terms considered ``unpopular''). For future studies, our findings can be applied to other models and further analysis of answer times in CQA. 

\section{Introduction}

    Community-driven online question and answer forums (CQA) are becoming widely used sources of information. These online platforms allow for anyone to ask or answer a question. It provides a place for people of all backgrounds and levels of expertise to provide input on a certain question. One popular CQA is Yahoo! Answers, featuring questions of all topics and expertise levels. On the other hand of the spectrum is Stack Overflow, the CQA that features strictly computer programming question and answers. Forums like these 2 receive around 40 million visits every month. These online question and answer forums are clearly becoming popular sources of information. 
    
    The CQA analyzed in this paper is iFixit's Answers forum. iFixit is a company founded in 2003 by two engineering students of the California Polytechnic State University in San Luis Obispo. What began as a business running out of the college dorms, is now a company that helps thousands of people fix their broken devices everyday, providing over 30,000 free repair guides with pictures and step-by-step instructions. iFixit also sells the specialized tools and parts needed for such repairs---tools like iPhone 7 battery adhesive strips or a Nexus 9 LCD digitizers. The main goal of this company is to provide people with the help and tools they might need to repair their devices, and in teaching people how to extend the lifetime of their devices, it allows people to save money and cut down on electronic waste.
    
    Along with repair guides and tool sales, another important component of this company is the online question and answer forum, Answers. This platform features questions related to device repair, featuring over 9,000 devices and over 100,000 solutions and answers. Questions on this forum range from broken devices like a jammed zippers to shattered iPhone screens. This forum is an integral part in teaching people how to fix their own devices. If people run into problems, they can always trust that they have a large community to turn to for help. Thus it's important, for both the individuals as well as iFixit, that askers get prompt answers. Fast response times will drive up user engagement and generate more traffic, which in turn is great for the reputation and longevity of Answers and iFixit. As such, investigation of response times would be both beneficial and informative. Analysis can reveal factors within the forum and from elements of the questions, that affect how quickly questions get answers. Suggestions for how users can ask questions to minimize answer times and for how the forum design can be improved, can be derived from such analysis. 
    
    However, to the best of our knowledge, analysis and prediction of answer times on forums has not been investigated by many researchers. A majority of the research focuses on assessing and predicting the question and answer quality. As such, there is need for further analysis of response times in these forums. This paper presents a survival analysis on the time until a question is answered on iFixit's Answers forum. It attempts to determine the factors that are significantly related to answer time, and create a cox proportional hazards model that can accurately predict the failure probability of a question.

\section{Related Work}

    In regards to investigation of forum response times, \citep{Bhat2014} developed a classification model to analyze response times of questions posted on Stack Overflow, and found that tag-based features like the number of tags included or the number of subscribers a certain tag has, were the best predictors of answer time. 

    \citep{Mamykina2011} found that the swift answer times of Stack Overflow's community is a result of the reputation system and the strict emphasis on factual and informative questions and answers, rather than discussion-based. 

    \citep{Asaduzzaman2013} analyzed unanswered questions on Stack Overflow to determine common characteristics and found that questions that went unanswered shared certain characteristics in that they were too short and vague, or utilized the tagging system incorrectly. 

\section{Materials}

    The data analyzed in this paper contained 7,760 questions posted from April 8, 2017 to July 7, 2017, the date the data was downloaded. We worked exclusively with questions in English. Of the questions in the data set, 4,951, or 63.8\%, recieved an answer. Questions that remained unanswered were considered right-censored. The time until event variable used in survival analysis methods was defined as the time until a question posted on the forum recieves an answer. Comments posted on the question are not considered answers, and an answer does not have to be accepted as the ``chosen solution'' to be considered the answer. We are only considering the time until the question recieves its first answer. The time until answer value for right-censored questions, questions that did not recieve an answer, is defined as the time from when the question was posted, to the time the data was downloaded. Figure a shows the distribution of answer times for all questions in the data set. 
    
    The following statistics were calculated from Kaplan-Meier estimates of survival probabilities for each question in the data. As Kaplan-Meier estimates adjust for the presence of censored observations, these statistics are adjusted for the presence of unanswered questions. In this analysis, survival is defined as the event that a question remains unanswered beyond a certain time, t. For questions in this data set, the mean survival time, or the average time until a question recieves its first answer is 775.75 hours, or 32.32 days. The median survival time, the time at which 50\% of the questions in the data recieved an answer is 9.16 hours. Table 1 provides more percentiles of survival time. 
    Figure A shows the Kaplan-Meier Curve for all questions in the data. The curve indicates that around 100 hours, the probability that a question recieves an answer after that time hovers around 0.35. The survival probability never reaches 0 due to the large amount of unanswered questions. 
    
    Variables created: 
    
    The data contained information about each question's title, text, and device, from which we created the following variables to use in the cox regression model. 

The time until a question recieves its first answer, as described above, was calculated as the time from the time the question was posted to the time the first answer was received in hours. If the question did not receive an answer, the time until answer value was calculated as the time from when the question was posted, to the time the data was downloaded, July 7, 2017 at 2:41 PM PDT. The censoring variable indicates 1 if the question received an answer, and 0 if the question remained unanswered by the time the data set was downloaded.

Categorical Variables: 

\begin{itemize}
  \item Questions were categorized based off of the devices it pertains to. Categories include: Apple Products, Android\/Other Phone, PC, Tablet, Electronics, Camera, Vehicle, Game Console, Home, Other. Home, Vehicle, Other. Table B shows the percentage of questions within each device category. 
  \item Whether or not the question's title contains at least one word that is considered ``frequently used'' among answered questions.
  \item Whether or not the question's title contains at least one word that is considered ``frequently used'' among unanswered question. 
  \item Whether or not the question's title ends in a question mark.
  \item Whether or not the question's title begins with a "Wh" word (e.g. "What", "Where"). 
  \item Whether or not the question's text contains any end punctuation marks (. ? !). 
  \item Whether or not the question's text is in all lower case. 
  \item Whether or not the asker updated the question's text (e.g., edited or added information) sometime after posting it. 
  \item Whether or not the asker included a greeting (e.g. ``Hello'', ``Greetings'') in the question's text.
  \item Whether or not the asker used polite language in the question's text (e.g., ``Thank you'', ``please''). 
  \item Whether or not the asker made an effort to solve the problem on their own, prior to asking the question.
  \item The day of the week the question was posted. 
  \item The time of day the question was posted. Times include: Morning, Afternoon, Evening, and Night. 
\end{itemize}

Numeric Variables:

\begin{itemize}
  \item The average frequency, or proportion of times a tag appeared in all of the data, for all of a question's user-defined tags. Questions without tags were assigned a value of 0 for this variable. 
  \item The average number of characters in each question's tags. 
  \item The number of characters in the question's text. 
  \item The number of characters in the user-defined device name. 
  \item The ratio of the number of newlines to the number of characters in the question's text.
\end{itemize}

\section{Methods} 

Univariate analyses for each variable of interest were performed on one of the training data sets. Variables with a p-value of less than 0.01 were entered into the final model. As the goal of this model is prediction, it was decided that all variables would be kept in the model to maintain predictive power. 

5-fold cross-validation was used to assess predictive performance. Training data sets contained 6,208 questions. Test data sets contained 1,552 questions. 

For each iteration of cross-validation, the model fit to the training data was used to calculate predicted hazard ratios from the corresponding test data. To assess prediction performance, those predicted ratios were entered into a separate cox regression model as the explanatory variable with the question's answer times as the survival time value. The resulting hazard ratio, R-square statistic, partial likelihood ratio and p-value, Somers' Dxy and Concordance statistic for that model were used as indicators of the prediction model's performance. If this model gave significant results, this would indicate that the predicted hazard ratio is signficantly associated with survival time, and that our prediction model performs well. Somers' Dxy and Concordance statistics are measures of how well the model can rank predictions. (add to this) The concept behind these metrics came from \cite{Chen}. These metrics were also computed for predictions on the training data itself. For each of the five iterations, there was no signficant difference between the metrics computed on the training data predictions, and the metrics computed on the test data. Table C shows average metrics for training and test data predictions. The metrics taken into consideration was the average of all metrics computed on test data. Although these metrics are not considerably high, they do not vary signicantly from training to test data sets.  

\section{Results and Discussion} 

The final model, including all variables described above, was fit to the full data set.
The proportional hazards assumption was assessed by using the cox.zph function of the survival package in R (expand on this, don't mention function name). This test is based on the correlation between scaled Schoenfeld residuals and a function of time. Signficant p-values of this test indicate a possible violation of the proportional hazard assumption. For our model, the ``Apple Product'' level of the device category variable had the lowest p-value of 0.01. The next lowest p-values hovered around 0.04, for variables like device\_length, and the Camera and Game Console levels of category. As Apple Product was the only major violation of this assumption, and stratifying on this variable substantially decreased the predictive power, this assumption was considered to be fine (rewrite this). 

Assessing the martingale residuals for each quantitative predictor indicated that the functional form appears to be adequate. Assessing the deviance residuals indicated that a considerable amount of questions were not predicted well by the model. Assessing the score residuals showed that some questions might have a strong influence on the fit of the model. However, in both cases, fitting the model without such questions resulted in worse predicted performance. 

The final model included all variables described above, along with the following transformations: 

\begin{itemize}
  \item Stratification on the time of day the question was posted. 
  \item Square root of the average frequency of a question's tags. 
  \item Quadratic polynomial on the length of the question's text. 
  \item Restricted cubic splines on the number of characters in the device name, the average length of a question's tags, and the ratio of newlines to number of characters in a question's text. Knots assigned to these variables were 5, 4, 4, respectively. 
\end{itemize}

The final model gave an AIC value of 69945.32. The partial loglikelihood ratio test statistic was 1307.15 with a p-value of 0. The R-squared value for this model is 0.1555. Variables that proved to be significant predictors with a p-value of less than 0.01 are: new_category (apple product, camera, game console, home, pc, and vehical), contain_unanswered, title_quesitonmark, update, prior_effort, avg_tag_score and the spline on avg_tag_length. Include some interpretations of the hazard ratio? 

While the model is significant overall, the metrics for the model's predictive accuracy are not impressive. This indicates that perhaps a different model might perform better in predicting the answer times for questions. Another suggestion is that a model with more quantitative variables might perform better as well.  


\section{Appendix} 

device

The data contained the original category variable as defined by iFixit. This category variable contained NAs (include how many), a result of the asker creating a question for a device not already on the website's data base. Questions that made the device clear were categorized accordingly. However there were a number of questions that did not explicitly declare the device their question pertained too. These questions were categorized as ``Other''. All Apple products (e.g., iPhones, iMacs, Apple watches) were given their own category. After pulling out iPhones from the original ``Phones'' category, the remaining phones were categorized as ``AndroidOther Phone''. Appliance and Household categories of the original variable were merged into ``Home''. ``Car and Truck'' were added to the ``Vehicle''. Lastly, any category that contained less than 2\% of the questions were categorized as ``Other''. All other categories remained the same as the original category. The final categories in the new categorization variable were Apple Product, Android/Other Phone, PC, Tablet, Electronics, Camera, Vehicle, Game Console, Home, and Other. 

contain answered and unanswered

  These variables were created based on the intuition that certain topics might be more popular, and even unpopular, among the answering community, and that questions concerning these topics might recieve an answer faster, or slower, than questions that don't. 
  
  To create these variables, the data was separated between answered and unanswered questions. Using text mining techniques, two lists were created-one for answered questions and another for unanswered questions--containing every word within the question's titles and the frequency or the number of time they occured among answered and unanswered questions. For ``frequently-used'' words in answered questions, that word would have to appear in more than 1\% of answered question's titles, and would have to appear in answered questions more than it appeared in unanswered questions. To determine the latter, a ratio of proportions was assessed. This ratio was calculated as the proportion of times a word occured among answered questions, to the proportion of times a word occured in unanswered questions. As an example, if ``cracked'' appeared in 2\% of answered questions and in 0.1\% of unanswered questions, it would be considered frequently used among answered questions as it occurs in more than 1\% of answered question's titles and occurs 20 times more in answered questions and in unanswered questions (ratio = 0.02/0.001 = 20). As for ``frequently-used'' words in unanswered questions, they must occur in 1\% or more of the question's titles and occur in more unanswered questions than unanswered questions (ratio < 1). Since there was some overlap between the words in each list and the device categories, every word that matched a device name was removed from the list. The resulting list for answered questions was 111 words, and 32 words in uanswered questions. Contain_answered and contain_unasnwered are logical variables, indicating true if a question's title contained at least one of the words in the frequently used words list for answered and unanswered questions, respectively. 
  
Title\_questionmark and title\_beginwh 

These two variables were created to determine if stating the title in the form of a question is associated with faster answer times. Title_questionmark is a logical variable indicating true if the question's title ends with a question mark. Title_beginwh is a logical variable indicating true if the title begins with a “wh” word (e.g. "what", "when"). These variables were created based off of the findings from \citep{Bhat2014}. 
  
  
Text\_contain\_punct

A logical variable indicating true if the question’s text contains any end punctuation marks (. ? !). This variable was created to investigate if run-on sentences, or sentences with no end punctuation, take longer to receive an answer. 

Prior\_effort 

Logical variable indicating true if the asker used words that indicate prior effort or research was done before asking the question (e.g. ``tried'', ``attempted'', ``tested''). This variable was created based off of findings from \cite{Bhat2014}. 

Ampm (change name)

The time of day the question was posted. If the question was posted between 5am and 12pm, it was categorized as ``morning''. If it was posted between 12pm and 5pm, it was categorized as ``afternoon'', between 5pm and 8am- ``evening'', and 8pm and 5 am, ``night''

Avg\_tag\_score

This variable was created to investigate the idea that some tags are more popular, or widely used than other, and that including such tags might increase the likelihood of that question recieving an answer. This variable is the average frequency, or proportion of times a question's tags appear in all of the data set. If a question has a higher average, than at least one of it's tags are frequently used. This variable was created based off of findings from \cite{Bhat2014}. 


Device\_length

The number of characters in the user-defined device variable. This variable was included to capture when a user incorrectly inputs the device name. For example, the device variable for question 390271 is, “Turtle Beach Ear Force Xmy grandson chewed through the wire while we was playing it's brand-new is there anyway I can have it fixedO One” and is 136 characters long. 11 questions in this data set also didn’t include any device. This variable was created based on the intuition that users who incorrectly define their devices make it harder for answerers to discern the topic of their question, and thus have longer answer times or might not recieve an answer.

Avg\_tag\_length

The average number of characters in each question's user-defined tags. If a question did not include a tag, this variable was set to 0. This variable, like the device-length variable, was created to capture when a user correctly or incorrectly used the tagging system. For example, question 390989 includes tag: “i need to repair the headset because i can not find the bluetoot” and is 64 characters long, while question 410254 includes tags "sound", "sound driver" and "speaker" and has an average tag length of 8 characters. It is hypothesized that users who use the tagging system correctly, as with the latter user, receive answers quicker than the former user. 

Newline\_ratio

This ratio of the number of newlines to the number of characters in the question's text. This variable was based on the intuition that question's that include newlines in the text are easier to read and thus have faster answer times. Questions with long text that also include newlines are generally easier to read than questions that don’t include any. 

\bibliographystyle{ECA_jasa}
\bibliography{questions}


\end{document}
